\documentclass[12pt]{article}
\usepackage{graphicx}
\usepackage{amsmath}
\usepackage{sbc-template}
%\usepackage{subfigure}
%\usepackage{times,amsmath,epsfig}
%\usepackage{graphicx,url}
\makeatletter
\newif\if@restonecol
\makeatother
\let\algorithm\relax
\let\endalgorithm\relax
\usepackage[lined,algonl,ruled]{algorithm2e}
\usepackage{multirow}
\usepackage[brazil]{babel}
\usepackage[utf8]{inputenc}

\sloppy

\title{Trabalho Prático 1: \\Interface de sockets e medição de desempenho}

\author{Pedro Lopes Miranda Junior}

\address{Departamento de Ciência da Computação -- Universidade Federal de Minas Gerais (UFMG) \email{plmj@dcc.ufmg.br}}

\begin{document}
\maketitle

\begin{resumo}
  O objetivo deste trabalho é (1) exercitar a interface de programação com sockets (em python),
  (2) se familiarizar com o ambiente virtualizado mininet para emulação de sistemas em rede e
  (3) medir e analisar o desempenho de aplicações simples neste ambiente.
  Para tanto, as seguintes etapas estão envolvidas:
  \begin{enumerate}
  \item Instalação do mininet e configuração de uma topologia.
  \item Implementação/execução de programas de teste no mininet e medição de desempenho.
  \item Análise de resultados e escrita do relatório.
  \end{enumerate}
\end{resumo}

\section{INTRODUÇÃO}
\label{introducao}

Assim, para testar os conceitos de redes de computadores, foi requisitado o teste de dois pares cliente/servidor
descritos abaixo.
\begin{enumerate}
\item[Par 1:] O cliente envia um certo número de mensagens para o servidor, sem paradas entre cada envio. 
Ao final, ele espera uma mensagem de um byte de volta do servidor.\\
\item[Par 2:] Cada vez que o cliente enviar uma mensagem, ele deverá esperar uma resposta de um byte do ser-
vidor. 
O cliente termina depois de fazer esse processo um certo número de vezes.
\end{enumerate}

\section{SOLUÇÃO PROPOSTA}
\label{solucao_proposta}

Para solucionar os problemas propostos foram feitos dois pares de cliente/servidor utilizando a linguagem \textit{Python}
e a biblioteca \textit{socket} para fazer a conexão do cliente com o servidor.
Outras bibliotecas usadas foram as \textit{sys, random} e a \textit{timeit}.\\
A primeira foi utilizada para recuperar os parâmetros passados na chamda do script. 
Já a segunda para criar as sequencias aleatórias de mensagens. A última 
para medir o tempo de execução da função \textit{client} em cada um dos pares.\\
Os demais detalhes de implementação serão tratados na subseção a seguir

\subsection{Par 1}
Para o par 1 era necessário criar um cliente que enviava mensagens intermitentes esperando ao fim de todas elas
uma resposta de 1 byte do servidor, e as premissas usadas em cada item do par estão descritas abaixo:

\subsubsection{Cliente}
O cliente recebe como parâmetro o número de mensagens e o tamanho de cada uma delas,
cria as mensagens aleatórias que servirão de envio para o servidor e as envia, uma a uma, 
esperando ao fim uma resposta de um byte. Para o funcionamento correto do par o cliente envia
ao fim das mensagens um sinal de fechamento da conexão apenas para envio, 
ficando de guarda quanto a resposta final do servidor, para só então fechar completamente a conexão.\\

A criação das mensagens foi feita separadamente do código do cliente para não interferir na análise de tempo
de envio das mensagens do cliente para o servidor. Para essa medição foi utulizada a função \textit{Timer} da biblioteca \textit{timeit}.
Ela calcula o tempo de execução de uma função específica chamando-a diversas vezes e tirando a média do tempo de execução da mesma. 
Nesse primeiro momento não há variação das mensagens, então essa média é a de envio de uma ou mais mensagens específicas. 
Além disso o procedimento citado é repetido  com o envio de outras mensagens, deixando a média ainda mais confiável. 
Para o teste cada mensagem foi enviada 10 vezes e foram criadas 4 mensagens diferentes para envio ao servidor.

\subsubsection{Servidor}

\subsection{Par 2}
\subsubsection{Cliente}
\subsubsection{Servidor}
Esses parâmetros são o número de mensagens a serem trocadas entre os pares e o tamanho de cada uma das mensagens.\\
Elas foram criadas numa função separada e anteriormente a troca de mensagens entre o cliente e o servidor para não
influenciar na medição do tempo.\\
\section{AVALIAÇÃO EXPERIMENTAL}
\label{avaliacao_experimental}

\subsection{Resultados}

\subsection{Análises}

\section{CONCLUSÃO}
\label{conclusao}

\end{document}
