\documentclass[12pt]{article}
\usepackage{graphicx}
\usepackage{amsmath}
\usepackage{sbc-template}
%\usepackage{subfigure}
%\usepackage{times,amsmath,epsfig}
%\usepackage{graphicx,url}
\makeatletter
\newif\if@restonecol
\makeatother
\let\algorithm\relax
\let\endalgorithm\relax
\usepackage[lined,algonl,ruled]{algorithm2e}
\usepackage{multirow}
\usepackage[brazil]{babel}
\usepackage[utf8]{inputenc}

\sloppy

\title{Trabalho Prático 1: \\Interface de sockets e medição de desempenho}

\author{Pedro Lopes Miranda Junior}

\address{Departamento de Ciência da Computação -- Universidade Federal de Minas Gerais (UFMG) \email{plmj@dcc.ufmg.br}}

\begin{document}
\maketitle

\begin{resumo}
  O objetivo deste trabalho é (1) exercitar a interface de programação com sockets (em python),
  (2) se familiarizar com o ambiente virtualizado mininet para emulação de sistemas em rede e
  (3) medir e analisar o desempenho de aplicações simples neste ambiente.
  Para tanto, as seguintes etapas estão envolvidas:
  \begin{enumerate}
  \item Instalação do mininet e configuração de uma topologia.
  \item Implementação/execução de programas de teste no mininet e medição de desempenho.
  \item Análise de resultados e escrita do relatório.
  \end{enumerate}
\end{resumo}

\section{INTRODUÇÃO}
\label{introducao}
Da internet a um sistema de comunicação entre uma estação aqui na terra e um robô em Marte,
redes de computadores é algo essencial em tempos onde a iformação é dada em tempo real.\\
Desse modo, as redes mais simples são compostas geralmente pela comunicação de um cliente com um servidor.
O cliente é o responsável pelo envio das mensagens. Por sua vez o servidor recebe as mensagens dos clientes
podendo ou não retornar uma resposta. \\

A comunicação efetiva entre esses pares depende, entre outras coisas da largura da banda de comunicação,
da distancia entre esses pares, do tamanho e quantidade das mensagens enviadas e do método utilizado para envio.\\

A largura de banda \ldots \\ %TODO Explicar
A distância entre o cliente e o servidor \ldots \\ %TODO Explicar
Já a quantidade e tamanho das mensagens envidas \ldots \\ %TODO Explicar
Por último o método ou protocólo de envio influencia diretamente \ldots \\ %TODO Explicar 

Há contudo modelo mais complexos de redes que serão abordados mais adiante na disciplina,
porém para uma iniciação aos conceitos de redes de computadores foi requisitado o teste de dois pares cliente/servidor
conforme descritos abaixo:
\begin{enumerate}
\item[Par 1:] O cliente envia um certo número de mensagens para o servidor, sem paradas entre cada envio.
Ao final, ele espera uma mensagem de um byte de volta do servidor.\\
\item[Par 2:] Cada vez que o cliente enviar uma mensagem, ele deverá esperar uma resposta de um byte do ser-
vidor.
O cliente termina depois de fazer esse processo um certo número de vezes.
\end{enumerate}

\section{SOLUÇÃO PROPOSTA}
\label{solucao_proposta}

Para solucionar os problemas propostos foram feitos dois pares de cliente/servidor utilizando a linguagem \textit{Python}
e a biblioteca \textit{socket} para fazer a conexão do cliente com o servidor.
Outras bibliotecas usadas foram as \textit{sys, random} e a \textit{timeit}.\\
A primeira foi utilizada para recuperar os parâmetros passados na chamda do script.
Já a segunda para criar as sequencias aleatórias de mensagens. A última
para medir o tempo de execução da função \textit{client} em cada um dos pares.\\
Os demais detalhes de implementação serão tratados na subseção a seguir

\subsection{Par 1}
Para o par 1 foi necessário criar um cliente que enviava mensagens intermitentes esperando ao fim de todas elas
uma resposta de 1 byte do servidor, e as premissas usadas em cada item do par estão descritas abaixo:

\subsubsection{Cliente}
O cliente recebe como parâmetro o número de mensagens e o tamanho de cada uma delas,
cria as mensagens aleatórias que servirão de envio para o servidor e as envia, uma a uma,
esperando ao fim uma resposta de um byte. Para o funcionamento correto do par o cliente envia
ao fim das mensagens um sinal de fechamento da conexão apenas para envio,
ficando de guarda quanto a resposta final do servidor, para só então fechar completamente a conexão.\\

A criação das mensagens foi feita separadamente do código do cliente para não interferir na análise de tempo
de envio das mensagens do cliente para o servidor.
Para essa medição foi utulizada a função \textit{Timer} da biblioteca \textit{timeit} e será evidenciada na análise experimental.


\subsubsection{Servidor}
Para esse par o servidor recebia ininterruptamente mensagens do cliente pareado com ele
e só fechava a conexão ao receber do cliente a instrução de que não receberia mais mensagem alguma.
O servidor então enviava uma mensagem de 1 Byte como resposta para o cliente e se encerrava.

\subsection{Par 2}
Já para o par 2 foi necessário implmentar um cliente que enviava mensagens ao servidor,
que respondia a cada uma recebida com uma nova mensagem de 1 Byte.
Após enviar todas as mensagen o cliente se encerrava, desconectando-se do servidor.

\subsubsection{Cliente}
O cliente recebe como parâmetro o número de mensagens e o tamanho de cada uma delas,
cria as mensagens aleatórias que servirão de envio para o servidor e as envia, uma a uma,
esperando uma resposta de um byte para cada mensagem enviada.
Para o funcionamento correto do par o cliente envia ao fim da última mensagens um sinal de fechamento da conexão,
avisando assim ao servidor o fim do envio.\\

A criação das mensagens foi feita separadamente do código do cliente para não interferir na análise de tempo
de envio das mensagens do cliente para o servidor.
Para essa medição foi utulizada a função \textit{Timer} da biblioteca \textit{timeit} e será mais detalhada na análise experimental.

\subsubsection{Servidor}
Nesse caso o servidor iniciava a conexão com seu cliente e a cada nova mensagem enviava uma mensagem de 1 byte como resposta.
Após o recebimento de todas as mensagens o mesmo esperava a mensagem de fechamento de conexão vindo do cliente.
Note que em momento algum o servidor sabe quantas mensagens o cliente enviará e o mesmo só encerra a conexão
quando recebe a mensagem do servidor de conexão fechada.
\section{AVALIAÇÃO EXPERIMENTAL}
\label{avaliacao_experimental}
Na avaliação experimental era necessário analisar a variação do tempo de envio das mensagens segundo a alteração da topologia da rede.
Para isso foi utilizada a máquina virtual do mininet. Essa máquina virtual emula elementos de rede como switches e hosts,
podendo inclusive alterar elementos como o tempo de comunicação entre eles ou a largura da banda.\\

Para o calculo do tempo de comunicação foi usada a função \textit{timeit} da biblioteca \textit{Timer} do python.
Ela calcula o tempo de execução de uma função específica chamando-a diversas vezes e tirando a média dos tempos de execução da mesma.
Nesse primeiro momento não há variação das mensagens, então essa média é a de envio de uma ou mais mensagens identicas.
Após isso o procedimento citado é repetido, mas dessa vez com o envio de mensagens diferentes.
Para o teste cada mensagem foi enviada 10 vezes e foram criadas 4 mensagens diferentes para envio ao servidor.

\subsection{Resultados}

\subsection{Análises}

\section{CONCLUSÃO}
\label{conclusao}

\end{document}
